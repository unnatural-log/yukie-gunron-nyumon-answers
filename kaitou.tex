\documentclass[dvipdfmx,uplatex,11pt]{jsarticle}
%
\usepackage[dvipdfmx]{graphicx}
\usepackage{amsmath,amssymb,amsthm}
\usepackage{enumitem}
\usepackage{wrapfig}
\usepackage{bm}
\usepackage{ascmac}
\setcounter{tocdepth}{2}
\usepackage{geometry}
\usepackage{framed}
\usepackage{latexsym}
%
\geometry{left=10mm,right=10mm,top=5mm,bottom=10mm}
%
\begin{document}
%
\section{集合論}
%
\subsection{1.1.6}
(1)
\begin{leftbar}
    \begin{proof}
        $c \in C$を任意にとる.$g$は全射であるから,$g(b)=c$をみたす$b \in B$が存在し,その$b$に対して$f$の全射性により,$f(a)=b$を満たす$a \in A$が存在する.
        以上の議論から,
        \[
            c = g(b)= g(f(a))=(g \circ f) (a)
        \]
        である,よって$g \circ f$も全射である.
    \end{proof}
\end{leftbar}
(2)
\begin{leftbar}
    \begin{proof}
       $a ,a' \in A$とし, $(g \circ f)(a)= (g \circ f) (a')$を仮定する.このとき,$g(f(a))=g(f(a'))$,$f(a),f(a') \in B$であり,このとき$g$の単射性から$f(a)= f(a')$となり,さらに$f$の単射性から$a=a'$となる.よって,
       \[
        (g \circ f)(a)= (g \circ f) (a') \Longrightarrow a=a'
       \]
       となるため,$g \circ f$も単射である.
    \end{proof}
\end{leftbar}
%
%
\newpage
%
\section{第2章・群の基本}

\subsection{2.1.1}
\noindent
\\
\textsl{Hint}:\\
\dotfill

\begin{leftbar}
\begin{proof}
まず,単位元について調べる.$G$は群であると仮定すると,単位元が存在し,それを$e$とすると,
\[
1 \circ e =1 ,\quad e \circ 1 =1
\]
であることから,単位元が存在すれば$e=1$である.\par
このもとで逆元を考察する.$ 0 \in G$について,ある$b \in G$が存在して,
\[
0 \circ b = 1,\quad b \circ 0 =1
\]
となる.しかし,このような$b \in G$は存在せず,矛盾.したがって$G$は群でない.
\end{proof}
\end{leftbar}

\newpage

\subsection{2.1.2}
\noindent
\\
\textsl{Hint}:\\
\dotfill

\begin{leftbar}
\begin{proof}
\noindent
単位元の存在について調べる.いま,単位元が存在すると仮定し,それを$e$とおく.この演算は可換であることに留意して,
\[
a \circ e =  a + e +ae =a
\]
とすると,
\[
e (a+1)=0
\]
よって,$e=0$または$a=-1$である.\par
まず,$a=-1$のとき,逆元が存在するか調べる.任意の$b \in \mathbb{R}$について,
\[
(-1) \circ b = -1 +b -b =-1 \ne 0
\]
よって,この演算によって$\mathbb{R}$は群とならない.よってただちに主張が従う.
\end{proof}
\end{leftbar}

\newpage

\subsection{2.3.5}
\noindent
\\
\textsl{Hint}:\\
\dotfill
\begin{leftbar}
    \begin{proof}
        $R_{>} =\{x \mid x \in \mathbb{R},0<x\}$とする.\\
        (1)\quad $\mathbb{R}^{\times}$の単位元$1$について,$1 \in \mathbb{R}_{>}$である.\\
        (2) \quad 演算の定義により,$\mathbb{R}_{>}$での加法は写像$\mathbb{R}_{>} \times \mathbb{R}_{>} \to \mathbb{R}_{>}$で定められるので,
        $x,y \in \mathbb{R}_{>}$のとき,$xy \in \mathbb{R}_{>}$である.\\
        (3) \quad $x \in \mathbb{R}_{>}$について,逆元は明らかに定義でき,$x^{-1} \in \mathbb{R}_{>}$である.\\
        (1),(2),(3)により,部分群の必要十分条件の3つが満たされ,$\mathbb{R}_{>}$は$\mathbb{R}^{\times}$の乗法についての部分群である.
    \end{proof}
\end{leftbar}

\end{document}


